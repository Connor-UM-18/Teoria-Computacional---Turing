\chapter{Conclusión}
La conclusión de desarrollar y solucionar el problema de la máquina de Turing es que se trata de un modelo computacional muy poderoso y versátil. Al implementar una máquina de Turing para resolver un problema específico, se obtienen varias conclusiones:\newline
\begin{enumerate}
    \item Universalidad: La máquina de Turing es capaz de simular cualquier algoritmo computacional, lo que demuestra su capacidad para resolver problemas teóricos y prácticos de manera general.\newline
    \item Complejidad: La máquina de Turing permite analizar la complejidad de un problema al medir la cantidad de operaciones necesarias para resolverlo. Esto es fundamental para evaluar la eficiencia de los algoritmos y comprender la viabilidad de las soluciones propuestas.\newline

    \item Representación abstracta: La máquina de Turing proporciona una abstracción poderosa para representar y modelar problemas complejos. Permite separar el concepto del problema en una cinta y un cabezal que interactúa con ella, lo cual facilita el diseño de soluciones.\newline
    
    \item Limitaciones y alcance: La máquina de Turing tiene sus limitaciones, especialmente en términos de la indecibilidad de algunos problemas o la imposibilidad de resolverlos de manera eficiente. Sin embargo, también tiene un amplio alcance y puede abordar una variedad de problemas teóricos y prácticos.\newline
\end{enumerate}

En conclusión general, el desarrollo y la solución de problemas utilizando la máquina de Turing permiten comprender y analizar la computabilidad y complejidad de los problemas. Además, brindan una base sólida para el diseño y análisis de algoritmos, y ayudan a investigar los límites de lo que se puede computar.

\\

\section{Complejidades}
La complejidad tanto espacial como temporal del algoritmo dependerá fielmente del tamaño de la entrada de ceros y unos y de las reglas de transición definidas en la máquina de Turing. Pero se espera que sea un comportamiento exponencial conforme el tamaño de la entrada incremente.