\chapter{Anexos}
\lstset{
    language=C,
    basicstyle=\ttfamily\small\color{black},
    numbers=left,
    numberstyle=\tiny,
    stepnumber=1,
    numbersep=8pt,
    backgroundcolor=\color{white},
    showspaces=false,
    showstringspaces=false,
    showtabs=false,
    frame=single,
    rulecolor=\color{magenta},
    tabsize=2,
    captionpos=b,
    breaklines=true,
    breakatwhitespace=false,
    title=\lstname,
    escapeinside={\%*}{*)},
    keywordstyle=\color{blue},
    commentstyle=\color{red},
    stringstyle=\color{orange},
    morecomment=[l][\color{red}]{\#},
    otherkeywords={=,!,<,>,*,+,-,&,|,^,~},
    numbers=left,                   % Coloca los números de línea a la izquierda
    numberstyle=\tiny\color{black}, % Estilo de los números de línea
    stepnumber=1,    % Incremento en el que se muestran los números de línea
    numbersep=8pt
}
\section{Código LATEX de este documento}
Dirección GitHub:\url{https://github.com/Connor-UM-18/Teoria-Computacional---Turing.git}\newline
Dirección Overleaf:\url{https://www.overleaf.com/2428945132mnyyfwyhddjq}\newline
\section{maquina\_turing.py}
Se presenta el código implementado para la solución al problema con extensión .py . Donde es necesario tener importada la librería random y pygame. \newline
\\
\begin{lstlisting}
#Teoria de la computacion
#Maquina de turing
#Alumno: Connor Urbano Mendoza
import random
import pygame

pygame.init() #Acceso al paquete pygame
#Ancho
WIDTH = 1300
#Altura
HEIGHT = 700
screen = pygame.display.set_mode((WIDTH,HEIGHT)) #Tamanio de ventana a imprimir
pygame.display.set_caption('Maquina de turing')
font = pygame.font.Font('freesansbold.ttf',20)#Tipo de fuente 1 del juego
big_font= pygame.font.Font('freesansbold.ttf',50)#Tipo de fuente 2 del juego
timer = pygame.time.Clock()#velocidad de actualizacion de nuestro juego a 60 fps
fps=60

#Variables e imagenes del juego
ubicacion = ['recuadro']

#Variables de turnos cambiantes
turn_step = 0
selection= 100
#Cargar imagenes en juego
cuadro = pygame.image.load('C:\\Users\\soyco\\OneDrive\\Documents\\ESCOM\\sem4\\Teoria\\P2\\turing\\img\\cuadro.png')
cuadro = pygame.transform.scale(cuadro,(101,101))

cuadrado = [cuadro]

lista_piezas = ['recuadro']

#ver variables/contador flash
boton_presionado = False

def dibujar_boton():
    boton_width = 150
    boton_height = 45
    boton_x = (WIDTH - boton_width) // 2
    boton_y = (HEIGHT - boton_height - 20)-100

    # Dibujar el botón como un rectángulo en la pantalla
    boton_rect=pygame.Rect(boton_x, boton_y, boton_width, boton_height)
    pygame.draw.rect(screen, (0, 255, 0), (boton_x, boton_y, boton_width, boton_height))
    texto = font.render("Siguiente", True, (0, 0, 0))
    texto_rect = texto.get_rect(center=(boton_x + boton_width // 2, boton_y + boton_height // 2))
    screen.blit(texto, texto_rect)
    return boton_rect
   
#Funcion para dibujar el contenido de la maquina de turing
def mostrar_transiciones(recorrido,estado):
    
    #recorrido="X,0,0,1,[1],1,1,1,"
    #estado="q2"
    cuadro_size = 100  # Tamaño de cada cuadro del tablero
    x = 0
    y = 255
    
    font = pygame.font.Font(None, 100)  # Fuente y tamaño del número de casilla

    transiciones = recorrido.split(',')  # Obtener las transiciones del recorrido

    for i, transicion in enumerate(transiciones):
        if i >= 12:
            break  # Salir del bucle si se han mostrado todas las transiciones posibles
        x = x + 100
        if transicion.startswith('['):
            # Realizar alguna acción especial para la transición que comienza con '['
            # Por ejemplo, cambiar el color del recuadro o agregar un marcador adicional
            numero_texto = font.render("^", True, 'yellow')  # Crear superficie de texto con la transición
            numero_rect = numero_texto.get_rect(center=((x, y+110)))  # Posición del texto en el centro del recuadro
            screen.blit(numero_texto, numero_rect)  # Pegar el texto en la pantalla
            numero_texto = font.render("|", True, 'yellow')  # Crear superficie de texto con la transición
            numero_rect = numero_texto.get_rect(center=((x, y+120)))  # Posición del texto en el centro del recuadro
            screen.blit(numero_texto, numero_rect)  # Pegar el texto en la pantalla
            numero_texto = font.render(estado, True, 'yellow')  # Crear superficie de texto con la transición
            numero_rect = numero_texto.get_rect(center=((x, y+190)))  # Posición del texto en el centro del recuadro
            screen.blit(numero_texto, numero_rect)  # Pegar el texto en la pantalla
            index=lista_piezas.index('recuadro')
            screen.blit(cuadrado[index],(x-50,y-55))
            
        numero_texto = font.render(transicion, True, 'black')  # Crear superficie de texto con la transición
        numero_rect = numero_texto.get_rect(center=((x, y)))  # Posición del texto en el centro del recuadro
        screen.blit(numero_texto, numero_rect)  # Pegar el texto en la pantalla
 
#Funcion para dibujar tablero
def dibujar_tablero():
    cuadro_size = 100  # Tamaño de cada cuadro del tablero
    tablero_width = 12 * cuadro_size  # Ancho total del tablero
    tablero_height = 1 * cuadro_size  # Altura total del tablero
    tablero_x = (WIDTH - tablero_width) // 2  # Posición X para centrar el tablero
    tablero_y = ((HEIGHT - tablero_height) // 2 )-200 # Posición Y para centrar el tablero

    for i in range(12):  # Iterar 12 veces para un tablero de 1x12
        columna = i % 12
        fila = 1
        x = tablero_x + columna * cuadro_size
        y = tablero_y + fila * cuadro_size
        if fila % 2 == 0:
            color = 'white' if columna % 2 == 0 else (226, 255, 255)
        else:
            color = (226, 255, 255) if columna % 2 == 0 else 'white'
        pygame.draw.rect(screen, color, [x, y, cuadro_size, cuadro_size])
        pygame.draw.rect(screen, 'black', [x, y, cuadro_size, cuadro_size], 2)  # Agregar borde de color 
       
 
# Función para generar automáticamente una cadena válida
def generate_auto_string():
    n = random.randint(0, 500)
    input_string = '0' * n + '1' * n
    return input_string

def turing_M (state = None, #estados de la maquina de turing
              blank = None, #simbolo blanco de el alfabeto dela cinta
              rules = [],   #reglas de transicion
              tape = [],    #cinta
              final = None,  #estado valido y/o final
              pos = 0):#posicion siguiente de la maquina de turing

    st = state
    if not tape: tape = [blank]
    if pos <0 : pos += len(tape)
    if pos >= len(tape) or pos < 0 : 
        print("Se inicializa mal la posicion")
        SystemExit(1)
    
    rules = dict(((s0, v0), (v1, dr, s1)) for (s0, v0, v1, dr, s1) in rules)
    """
        Estado	Símbolo leído	Símbolo escrito	       Mov. 	Estado sig.
        qn(s0)  1,0,X,Y,B(v0)	  1,0,X,Y,B(v1)       R o L(dr)	   qn(s1)
    """
    while True:
        with open('C:\\Users\\soyco\\OneDrive\\Documents\\ESCOM\\sem4\\Teoria\\P2\\turing\\output\\turing.txt', 'a') as archivo:
            archivo.write(st+ '\n')
            #print (st, '\t', end=" ")
            for i, v in enumerate(tape):
                if i==pos: 
                    #print ("[%s]"%(v,),end=" ")
                    archivo.write('['+v+'],')
                else: 
                    #print (v, end=" ")
                    archivo.write(v+',') 
            #print()
            archivo.write('\n')
            if st == final: 
                print("Cadena valida.")
                break
            if (st, tape[pos]) not in rules: 
                print("Cadena invalida.")
                break
            
            (v1,dr,s1) = rules [(st, tape[pos])]
            tape[pos]=v1 #rescribe el simbolo de la cinta
    
    #movimiento del cabezal
        if dr == 'left':
            if pos > 0: pos -= 1
            else: tape.insert(0, blank)
        if dr == 'right':
            pos += 1
            if pos >= len(tape): tape.append(blank)
        st = s1

def turing_MG (state = None, #estados de la maquina de turing
              blank = None, #simbolo blanco de el alfabeto dela cinta
              rules = [],   #reglas de transicion
              tape = [],    #cinta
              final = None,  #estado valido y/o final
              pos = 0):#posicion siguiente de la maquina de turing

    st = state
    if not tape: tape = [blank]
    if pos <0 : pos += len(tape)
    if pos >= len(tape) or pos < 0 : 
        print("Se inicializa mal la posicion")
        SystemExit(1)
    
    rules = dict(((s0, v0), (v1, dr, s1)) for (s0, v0, v1, dr, s1) in rules)
    """
        Estado	Símbolo leído	Símbolo escrito	       Mov. 	Estado sig.
        qn(s0)  1,0,X,Y,B(v0)	  1,0,X,Y,B(v1)       R o L(dr)	   qn(s1)
    """
    while True:
        with open('C:\\Users\\soyco\\OneDrive\\Documents\\ESCOM\\sem4\\Teoria\\P2\\turing\\output\\turing.txt', 'a') as archivo:
            archivo.write('|-'+st+ '->')
            #print (st, '\t', end=" ")
            for i, v in enumerate(tape):
                if i==pos: 
                    #print ("[%s]"%(v,),end=" ")
                    archivo.write('['+v+'],')
                else: 
                    #print (v, end=" ")
                    archivo.write(v+',') 
            #print()
            if st == final: 
                print("Cadena valida.")
                break
            if (st, tape[pos]) not in rules: 
                print("Cadena invalida.")
                break
            
            (v1,dr,s1) = rules [(st, tape[pos])]
            tape[pos]=v1 #rescribe el simbolo de la cinta
    
    #movimiento del cabezal
        if dr == 'left':
            if pos > 0: pos -= 1
            else: tape.insert(0, blank)
        if dr == 'right':
            pos += 1
            if pos >= len(tape): tape.append(blank)
        st = s1
        
with open('C:\\Users\\soyco\\OneDrive\\Documents\\ESCOM\\sem4\\Teoria\\P2\\turing\\output\\turing.txt', 'w') as archivo:
    pass
print("Maquina de turing")
print("\n--- Menu ---")
print("1. Ingresar una cadena manualmente")
print("2. Generar una cadena automaticamente")
print("0. Salir")

option = input("Elige una opcion: ")

if option == '1':
    input_string = input("Ingrese una cadena menor a 11 caracteres. \nDigite la cadena deseada:  ")
    #se puede cambiar las reglasde transicion para otra maquina de turing
    turing_M (state = 'q0', #estado inicial de la maquina de turing
                blank = 'B', #simbolo blanco de el alfabeto dela cinta
                tape = list(input_string),#inserta los elementos en la cinta
                final = 'q4',  #estado valido y/o final
                rules = map(tuple,#reglas de transicion
                            [
                            "q0 0 X right q1".split(),
                            "q0 Y Y right q3".split(),
                            "q1 0 0 right q1".split(),
                            "q1 1 Y left q2".split(),
                            "q1 Y Y right q1".split(),
                            "q2 0 0 left q2".split(),
                            "q2 X X right q0".split(),
                            "q2 Y Y left q2".split(),
                            "q3 Y Y right q3".split(),
                            "q3 B B right q4".split(),
                            ]   
                            )
                )  
    #Animamos la maquina de turing
    #lista_estados = [int(num) for num in recorrido.split(",")]
    coordenadas=(235,85)
    nueva_coordenada=(0,0)
    run=True
    contador=1
    leerlinea=0
    with open('C:\\Users\\soyco\\OneDrive\\Documents\\ESCOM\\sem4\\Teoria\\P2\\turing\\output\\turing.txt', 'r') as archivo:
        lineas = archivo.readlines()
    
    while run:
        if len(lineas) == leerlinea:
            print("Programa termino.")
            SystemExit(0)
        timer.tick(fps)
        # Rellenar la pantalla con el color
        screen.fill((30, 22, 37))
        dibujar_tablero()
        boton_rect = dibujar_boton()
        estado = lineas[leerlinea].strip()  # Eliminar espacios en blanco al inicio y al final de la línea
        recorrido = lineas[leerlinea+1].strip()
        mostrar_transiciones(recorrido,estado)
        for event in pygame.event.get():
            if event.type == pygame.QUIT:
                run = False

            if event.type == pygame.MOUSEBUTTONDOWN and event.button == 1:
                mouse_pos = pygame.mouse.get_pos()
                if boton_rect.collidepoint(mouse_pos):  # Verificar si se hizo clic en el botón
                    leerlinea=leerlinea+2
        pygame.display.flip()
    pygame.quit()
  

elif option == '2':
    input_string = generate_auto_string()
    print("Cadena generada automaticamente:", input_string)
    print("Tamanio de cadena:", str(len(input_string)))
    turing_MG (state = 'q0', blank = 'B', tape = list(input_string),final = 'q4', rules = map(tuple,
                            #reglas de transicion
                            [
                            "q0 0 X right q1".split(),
                            "q0 Y Y right q3".split(),
                            "q1 0 0 right q1".split(),
                            "q1 1 Y left q2".split(),
                            "q1 Y Y right q1".split(),
                            "q2 0 0 left q2".split(),
                            "q2 X X right q0".split(),
                            "q2 Y Y left q2".split(),
                            "q3 Y Y right q3".split(),
                            "q3 B B right q4".split(),
                            ]   
                            )
                )
    
    
else:
    print("Opcion invalida. Inténtalo de nuevo.")
    SystemExit(0)

print("Programa termino.")

\end{lstlisting}